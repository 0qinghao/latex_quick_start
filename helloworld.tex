\documentclass[UTF8]{ctexart}
\usepackage{booktabs}
\usepackage{graphicx}
\begin{document}

\tableofcontents

\newpage
\section{练习}
Hello World !
% paragraph 和 subparagraph 几乎不用

special characters: \$ \% \& \#

Chinese characters: 林,中、日、韩字符。

\subsection{两种列表}
列表的关键字:itemize 和 enumerate。
\subsubsection{无序列表}
\begin{itemize}
    \item 无序列表1
    \item 无序2
    \item itemize
\end{itemize}
\subsubsection{有序列表}
\begin{enumerate}
    \item 有序列表1
    \item 有序2
    \item enumerate
\end{enumerate}

\newpage
\section{正文}
正文从这里\footnote{\label{fntest}Hello footnote}开始。测试引用 \ref{fntest} 效果。

\newpage
\section{表格}
默认表格测试。
\begin{table}[htb]
    \centering
    \caption{期望值}
    \label{tab:E}
    \begin{tabular}{c|c||c}
        % c l r 表示对齐方式
        1 & 2 & 3\\
        \hline
        a & b & c\\
    \end{tabular}
\end{table}

\newpage
\section{图像}
默认图像添加,需要宏包 graphicx。
\begin{figure}[htb]
    \centering
    \caption{分布情况}
    \includegraphics[width=5 cm]{test.jpg}
    \label{fig:dist}
\end{figure}

图 \ref{fig:dist} 展示了分布情况。

\newpage
\section{插入代码}
似乎不需要另外包含宏包。
\begin{verbatim}
import util.*
function foo = myFun(a, b, c)
    if a == '{' || a == '['
        N = norm(a .* b - c)
        % REMOVE ADDITIONAL SPACES
        foo = -N * a(3) / N
        p = foo^N - 17
        r = 42/0.8e15
        d = 4.7e11
        neg = -r
    end

    try something; catch e; end

    for k = 1:N
        % ADD INDENTATION AFTER LINE BREAK
        k++

        t = a * k ...
            + b .* k^2 ... % comment
            + c * k^3
        vectorofstrings = ['α' 'β' 'γ'];
        vectorofstrings = ['α', 'β', 'γ'];
        vectorofstuff = ['foo' -dead('beef', 3.14, bar) -foo('bar', '42')]

        if (norm(t))% ADD NEWLINE BEFORE AND AFTER BLOCK
            fprintf('Hello world \n');
        end
    end
end
\end{verbatim}

用 listings 宏包似乎可以更轻松地处理插入代码的工作,暂时应该不会用到插入代码的功能,暂时不深入学习。

\newpage
\section{分割 section 测试 1}
第一段。

第二段。
\newpage
\section{分割 section 测试 2}
分离文件 2 第一段。

分离文件 2 第二段。

\end{document}